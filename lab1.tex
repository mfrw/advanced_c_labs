%author : mfrw
\documentclass[9pt]{article}


\begin{document}

\title{Lab I}
\author{C Refresher Module}

\maketitle

\section{FS Walker}
Consider a Unix system where data is stored in directories. There is a root directory \textbf{(denoted by ‘/’)} which contains several other directories, each with a unique name. 
These directories may further contain directories and so on, thus forming a hierarchy. A directory can be uniquely identified by its name and its parent directory (the directory 
it is directly contained in). This is usually encoded in a path, which consists of several parts each preceded by a forward slash \textbf{(`/')}. The final part is the name of the directory, 
and everything else gives the path of its parent directory. \\
Example: \textit{\textbf{`/home/mfrw/linux'}} is the path to the directory \textit{\textbf{`linux'}}. \\

You are given a file \textit{\textbf{entries.in}} as per the format:
\begin{itemize}
	\item First set of lines consist of directory path in each line existing on a Unix computer.
	\item The second set of lines, as separated by an empty line from the first set, consists of a directory path in each line that you want to test for its existence.
		\em{Assume all the entries are absolute, i.e start with \textbf{`/'}} \em
\end{itemize}
\textbf{Your task is to:}
\begin{itemize}
	\item Build a directory tree from the first set of lines using an \em{appropriatae} \em data structure where each node contains the name of the directory and has a pointer to the child directory nodes.
	\item Use the directory tree to check if the paths given in the second set of lines exist or not.\\
		Print \textit{\textbf{Yes/No}} on the console, corresponding to each of the path to be tested.
\end{itemize}
\renewcommand{\thefootnote}{\fnsymbol{footnote}}
\textbf{Submission:} Upload on backpack:
\begin{itemize}
	\item A zipped\footnote[2]{We will run a plagiarism detector.} file including -- a \textbf{makefile} to compile the program\footnote[3]{We expect a C program and no other language.},
	a simple one paragraph writeup (\textbf{readme}) of the approach used,
		the \textbf{source} code of the program \& the \textit{\textbf{entries.in}} file.
\end{itemize}



%\textbf{For those who want more} \\
%Any of these or all would make you eligible for a bonus:
%\begin{itemize}
%	\item Try solving the same using a \textit{\textbf{B Tree/B+ Tree}}
%	\item Use a generic linked list implementation. \em{See the linux kernel implementation of linked lists} \em
%\end{itemize}

\end{document}
