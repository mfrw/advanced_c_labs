%author : mfrw
\documentclass{article}
\title{\vspace{-7cm}}
\usepackage{listings}
\usepackage{epigraph}
\usepackage{amsmath}


\begin{document}

\title{Lab III}
\author{C Refresher Module}

\maketitle
\epigraph{\em{``Today's lab is difficult as our lives. First day of lab invokes my nihilism.''\em}}
%nihilism = $\displaystyle \frac{existence}{esistence} \times 0$}
{\texttt{-Anonymous}}
\section{Matrix Multiplication}
\renewcommand{\thefootnote}{\fnsymbol{footnote}}
The goal of this lab is to implement \texttt{Matrix Multiplication} using threads. This computation comes in a class of problems called as \texttt{embarrassingly parallel}. It means the differnt operations in this problem are indipendent of each other, so easily parallelizable.
\\
You are given two matrices, A and B, where matrix A contains M rows and K columns and matrix B contains K rows and N columns, the matrix product of A and B is matrix C, where C contains M rows and N columns.
The input is from \texttt{stdin} and as follows:
\begin{itemize}
	\item The first line contains space seperated values of \textbf{M \& K}.
	\item The next \textbf{M} lines contain \textbf{K} space seperated values, followed by a blank line.
	\item The following line contains space seperated values of \textbf{K \& N}.
	\item The next \textbf{K} lines contain \textbf{N} space seperated values.
\end{itemize}
\textbf{Your task is to:}
\begin{itemize}
	\item Build a  \textit{\textbf{multi-threaded}} program that does matrix multiplication.
	\item Create a \textbf{\textit{thread}} for each element of the resultant matrix \textbf{C}.
	\item Print the resultant Matrix.
\end{itemize}
\newpage
\textbf{Submission\footnote[2]{We will use a plagiarism detector.}}: Upload a zipfile on backpack including: \\ 
\begin{itemize}
	\item The \textbf{source code\footnote[3]{Please \textbf{\textit{`indent -linux matmul.c'}} before submitting.}} of the program.
	\item The \textbf{makefile} for compilation.
	\item A small \textbf{write-up} of the approach used.
\end{itemize}


\section{Example}
\begin{verbatim}
	mfrw@kp:-$ cat test.in
	3 4
	97 58 35 79
	43 73 88 32
	74 66 99 89

	4 5
	43 64 16 83 46
	64 37 67 17 79
	92  8  8 82 63
	37 55 96 92 94

	mfrw@kp:-$ ./matmul < test.in
	14026 12979 13302 19175 18675
	15801  7917  9355 14970 16297
	19807 12865 14942 23570 23221
	mfrw@kp:-$ 
\end{verbatim}
\newpage
\section{Hints}
\em{Please dont generate random numbers by hand. Modify the program below to make a random matrix.} \em
%\begin{lstlisting}[language=C]
\begin{verbatim}
#include<stdio.h>
#include<time.h>

int main(int argc, char *argv[]) {
    srand(time(NULL));
    int size = 10;
    int mod_num = 100;
    if (argc == 2)
        size = atoi(argv[1]);
    while (size--)
        printf("%2d ", (int)random()%mod_num);
    printf("\n");
    return 0;
}

\end{verbatim}
%\end{lstlisting}
\section{For those who want more}
\texttt{Happy New Year !!!} \\
It's for \textit{\textbf{you}} and us too. Finish the lab and off you go to enjoy.
\section{I still want more}
\textbf{Have a go at this:}
\begin{itemize}
	\item Your program should take a \textit{\textbf{command-line argument}} the number of threads to use for the computation.
	\item \textit{Appropriately} break-down the work equally to the threads created.
	\item If no command-line argument is given you should run threads equal to number of \textit{\textbf{cores\footnote[2]{\texttt{sysconf(\_SC\_NPROCESSORS\_ONLN)}}}} on your machine. \texttt{(man sysconf)}
\end{itemize}
\end{document}
